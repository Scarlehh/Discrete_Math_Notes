\documentclass[../main.tex]
		
		\begin{document}
			\section{Circuits}
	\begin{description}
		\item[Task:] Use closed walks to understand the structure of graphs better.
		\item[Definition:] Let $(V, E)$ be a graph. A walk $v_0 v_1 \dots v_n$ in $(V, E)$ is called \underline{closed} if $v_0 = v_n$, \textbf{i.e.} if it starts and edges at the same vertex.
		\item[Definition:] Let $(V, E)$ be a graph. A \underline{circuit} is a nontrivial closed trail in $(V, E)$, \textbf{i.e.} a closed walk with no repeated edges passing through at least two vertices.
		\item[Definition:] A circuit is called \underline{simple} if the vertices $v_0, v_1, v_2, \dots v_{n-1}$ are distinct.
		\item[NB:] This is the strangest condition regarding vertices tha we can impose ince $v_0 = v_n$.
		\item[Alternative terminology:] Some authors use \underline{cycle} to denote a simple circuit, which for ohers \underline{cycle} denotes a circuit regardless of whether it is simple or not.
		\item[Q:] When does a graph have simple circuits?
		\item[A:] We can give 2 criteria for the existence of simple circuits:
		\begin{enumerate}
			\item Every vertex has degres $\geq$ 2.
			\item $\forall u, v \in V$ s.t. $\exists$ 2 distinct paths from $u$ to $v$.
		\end{enumerate}
		\item[Theorem:] If $(V, E)$ has no isolated or pendant vertices, \textbf{i.e.} $\forall v \in V$ deg $v \geq$ 2, then $(V, E)$ contains at least one simple circuit.
		\item[Proof:] Consider all paths $(V, E)$. The maximum length of a path is $\#(V) - 1$ since a path of length $p$ passed through $p+1$ vertices. Take a path $v_0 v_1 \dots v_m$ is $(V, E)$ of maximum length, \textbf{i.e.} any other path in $(V, E)$ has length $\leq m = $ length of $v_0 v_1 \dots v_m$. Now consider the vertex $v_m$. deg $v_m \geq$ 2 by assumption. We know $v_{m-1}$ is adjacent to $v_m$ since the edge $v_{m-1} v_m$ is part of the path $v_0 v_1 \dots v_m$, but deg $v_m \geq $ 2 means $\exists w \in V$ s.t. $ww_m \in E$. If $w \neq v_i$ for $0 \leq i \leq m-2$, then $v_0 v_1 \dots v_m w$ is a path in $(V, E)$ longer than $v_0 v_1 \dots v_m \Rightarrow \Leftarrow$ to the act that $v_0 v_1 \dots v_m$ was chosen of maximal length. Therefore, $w = v_i$ for some $0 \leq i \leq m-2$, but then $v_i v_{i+1} \dots v_m v_i$ is a simple circuit in the graph.
		\item[qed]
		\item[Theorem:] Let $(V, E)$ be an undirected graph and let $u, v \in V$ be vertices s.t. $u \neq v$ and $\exists$ at least two distinct paths in $(V, E)$ from $u$ to $v$. Then the graph contains at least one simple circuit.
		\item[Proof:] Let $a_0 a_1 a_2 \dots a_m$ and $b_0 b_1 \dots b_n$ be the two distinct paths in the graph between $u$ and $v$, \textbf{i.e.} $a_0 = b_0 = u$ and $a_m = b_m = v$. Let $m \leq n$. Since the paths are distinct $\exists i$ with $0 \leq i \leq m$ s.t. $a_1 \neq b_i$. Choose the smallest $i$ for which $a_i \neq b_i$, \textbf{i.e.} $a_0 = b_0, a_1 = b_1, \dots, a_{i-1} = b_{i-1}$, but $a_i \neq b_i$. We have thus eliminated the redundencies at the beginning of the paths. We now need to eliminate redundencies at the other end of the paths. We know $a_m = b_m$ so $a_j \in \{b_m \mid i-1 < k \leq n \}$ is certainly satisfied for $j=m$, but we want to choose the \underline{smallest} index for which this condition is satisfied. Let this index be $p \Rightarrow a_p \in \{b_k \mid i-1 < k \leq n \}$, \textbf{i.e.} $a_p = b_s$ for some $s$ s.t. $i-1 < s \leq n$. Since $p$ is the smallest index satisfying $a_p \in \{b_k \mid i-1 < k \leq n \}, a_i, a_{i+1}, \dots, a_{p-1} \notin \{b_k \mid i-1 < k \leq n \} \Rightarrow \underset{indices \: running \: in \: increasing \: order}{\underbrace{a_{i-1} a_i \dots a_p}} (= b_s) \underset{indices \: running \: in \: decreasing \: order}{\underbrace{b_{s-1}\dots b_i}} a_{i-1} (=b_{i-1})$ is a simple circuit in $(V, E) \Rightarrow (V, E)$ has at least one simple circuit.
		\item[qed]
	\end{description}
	

\end{document}