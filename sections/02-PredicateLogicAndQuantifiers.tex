\documentclass[../main.tex]
		
		\begin{document}
			\section{Predicate logic and Quantifiers}
	\begin{description}
		\item[Task:] Understand enough predicate logic to make sense of quantified statements.
		\item In predicate logic, propositions depend on variable x, y, z, so their truth value may change depending on which values these variables assume: $P(x), Q(x, y), R(x, y, z)$
	\end{description}
	
	\subsection{Introduce quantifiers}
	\subsubsection{$\exists$ existential quantifier}
	\begin{description}
		\item[Syntax:] $\exists xP(x)$
		\item[Definition:] $\exists xP(x)$ is true if $P(x)$ is true or some value of $x$; it is false otherwise.
	\end{description}
	
	\subsubsection{$\forall$ universal quantifier}
	\begin{description}
		\item[Syntax:] $\forall xP(x)$
		\item[Definition:] $\forall xP(x)$ is true if $P(x)$ is true for all allowable values of $x$. It is false otherwise.
	\end{description}
	
	\subsubsection{$\exists!$ for one and only one}
	\begin{description}
		\item[Syntax:] $\exists! xP(x)$
		\item[Definition:] $\exists! xP(x)$ is true if $P(x)$ is true for exactly one value of $x$ and false for all often values of $x$; otherwise, $\exists! xP(x)$ is false.
	\end{description}
	
	\subsection{Alternation of Quantifiers}
	\begin{tabular}{lr}
		$\forall x\exists y\forall z$ & $P(x, y, z)$ \\
	\end{tabular}
	~\\
	\textbf{NB:} The order \underline{cannot} be exchanged as it might modify the truth values of the statement (think of examples with two quantifiers).
	
	\subsection{Negation of Quantifiers}
	\begin{tabular}{rcl}
		$\lnot(\exists xP(x))$ & $\leftrightarrow$ & $\forall x\lnot P(x)$ \\
		$\lnot(\forall xP(x))$ & $\leftrightarrow$ & $\exists x\lnot P(x)$ \\
	\end{tabular}
	

\end{document}