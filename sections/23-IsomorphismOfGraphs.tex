\documentclass[../main.tex]
		
		\begin{document}
			\section{Isomorphism of Graphs}
	\begin{description}
		\item[Definition:] An \underline{isomorphism} between two graphs $(V, E)$ and $(V',E')$ is a bijective function $\varphi : V \rightarrow V'$ satisfying that $\forall a, b \in V$ with $a \neq b$ the edge $ab \in E \Leftrightarrow$ the edge $\varphi (a) \varphi (b) \in E'$.
		\item[Recall:] A function $\varphi : V \rightarrow V'$ is bijective $\Leftrightarrow$ it has an inverse $\varphi^{-1} : V' \rightarrow V'$. \\
		The bijection $\varphi V \rightarrow V'$ that gives the isomorphism between $(V, E)$ and $(V' E')$ thus sets up the following:
		\begin{enumerate}
			\item A 1-1 correspondence of the vertices $V$ of $(V,E)$ with the vertices $V'$ of $(V',E') \rightarrow$ comes from $\varphi: V \rightarrow V'$ being bijective.
			\item A 1-1 correspondence of the edges $E$ of $(V,E)$ with the edges $E'$ of $(V',E') \rightarrow$ comes from the additional property in the definition of an isomorphism that $\forall a, b \in V$ with $a \neq b, ab \in E \Leftrightarrow \varphi (a) \varphi (b) \in E'$. 
		\end{enumerate}
		\item[Remark:] Just like an isomorphism of groups discussed earlier in the course, an ismorphism of graphs means $(V,E)$ and $(V',E')$ have the same "iso" from "morph\'{e}". "Being isomorphic" is an equivalence relations, so we get classes of graphs that have the same form as our equivalence classes.
		\item[Definition:] If there exists an isomorphism $\varphi :V \rightarrow V'$ between two graphs $(V, E)$ and $(V', E')$, then $(V, E)$ and $(V', E')$ are called isomorphic.
	\end{description}
	

\end{document}