\documentclass[../main.tex]
		
		\begin{document}
			\begin{description}
		\item[Task:] Define the most useful kind of relation.
		\item[Definition:] A relation $R$ on a set $A$ is called
		\begin{enumerate}
			\item \underline{reflexive} iff (if and only if) $\forall x \in A, xRx$
			\item \underline{symmetric} iff $\forall x, y \in A, xRy \rightarrow yRx$
			\item \underline{transitive} iff $\forall x, y, z \in A, xRy \land yRz \rightarrow xRz$
		\end{enumerate}
		\item An equivalence relation on $A$ is a relation that is reflexive, symmetric and transitive.
		\item[Notation:] Instead of $xRy$, an equivalence relation is often denoted by $x \equiv y$ or $x \sim y$.
		\item[Examples:]
		\begin{enumerate}
			\item[]
			\item "=" equality is an equivalence relation.
			\begin{enumerate}
				\item $x=x$ reflexive
				\item $x=y \Rightarrow y=x$ symmetric
				\item $x=y \land y=z \Rightarrow x=z$ transitive 
			\end{enumerate}
			\item $A = \mathbb{N} \\
			x \equiv y$ mod $3$ is an equivalence relation. $x \equiv y$ mod 3 means $x-y = 3m$ for some $m \in \mathbb{Z}$, \textbf{i.e.} $x$ and $y$ have the same remainder when divided by 3. The set of all possible remainders is $\{0, 1, 2\}$ \\
			\textbf{NB:} In correct logic notation, $x \equiv y$ mod 23 if $\exists m \in \mathbb{Z} s.t. x-y=3m$
			\begin{enumerate}
				\item $x \equiv x$ mod 3 since $x-x=0=3 \times 0 \rightarrow$ reflexive
				\item $x \equiv y$ mod 3 $\Rightarrow y \equiv x$ mod 3 because $x \equiv y$ mod 3 means $x-y=3m$ for some $m \in \mathbb{Z} \Rightarrow y-x=-3m=3 \times (-m) \Rightarrow y \equiv x$ mod 3 $\rightarrow$ symmetric
				\item Assume $x \equiv y$ mod 3 and $y \equiv z$ mod 3 \\
				$x \equiv y$ mod 3 $\Rightarrow \exists m \in \mathbb{Z}$ s.t. $x-y=3m \Rightarrow y=x-3m$ \\
				$y \equiv z$ mod 3 $\Rightarrow \exists p \in \mathbb{Z}$ s.t. $y-z=3p \Rightarrow y=z+3p$ \\
				Therefore, $x-3m=z+3p \Leftrightarrow x-z=3p+3m = 3(p+m)$ \\
				Since $p, m \in \mathbb{Z}, p+m \in \mathbb{Z} \Rightarrow x \equiv z$ mod 3 $\rightarrow$ transitive.
			\end{enumerate}
			\item Let $f:A \rightarrow A$ be any function on a non empty set A. We define the relation $R = \{(x, y) \mid f(x)=f(y) \}$
			\begin{enumerate}
				\item $\forall x \in A, f(x)=f(x) \Rightarrow (x, x) \in R \rightarrow reflexive$
				\item If $(x, y) \in R$, then $f(x)=f(y) \Rightarrow f(y)=f(x)$, \textbf{i.e.} $(y, x) \in R \rightarrow$ symmetric
				\item If $(x, y) \in R$ and $(y, z) \in R$, then $f(x)=f(y)$ and $f(y)=f(z)$, which by the transitivity of equality implies $f(x)=f(z)$, \textbf{i.e.} $(x, z) \in R$ as needed, so $R$ is transitive as well.\\
				$f(x)$ can be $e^x, sin\:x, (x)$, etc.
			\end{enumerate}
			\pagebreak
			\item Let $\lambda$ be the set of all triangles in the plane. $ABC \sim A'B'C'$ if $ABC$ and $A'B'C'$ are similar triangles, \textbf{i.e.} have equal angles.
			
			\begin{figure}[t!]
				\centering
				\begin{tikzpicture}
					\coordinate (A) at (0, 0);
					\coordinate (B) at (1, 1);
					\coordinate (C) at (2, 0);
					\draw (A)--(B)--(C)--cycle;

					\tkzLabelAngle[pos = 0.4](B,A,C){A}
					\tkzLabelAngle[pos = 0.4](A,B,C){B}
					\tkzLabelAngle[pos = 0.4](A,C,B){C}
					
					\coordinate (A') at (3, 0);
					\coordinate (B') at (5, 2);
					\coordinate (C') at (7, 0);
					
					\draw (A')--(B')--(C')--cycle;
					
					\tkzLabelAngle[pos = 0.5](B',A',C'){A'}
					\tkzLabelAngle[pos = 0.5](A',B',C'){B'}
					\tkzLabelAngle[pos = 0.5](A',C',B'){C'}
				\end{tikzpicture}
			\end{figure}
			
			\begin{enumerate}
				\item $\forall ABC \in \lambda, ABC \sim ABC$ so $\sim$ is reflexive
				\item $ABC \sim A'B'C' \Rightarrow A'B'C' \sim ABC$ so $\sim$ is symmetric
				\item $ABC \sim A'B'C'$ and $A'B'C' \sim A''B''C'' \Rightarrow ABC \sim A''B''C''$, so $\sim$ is transitive
				\item[Clearly] (a), (b), (c) use the fact that equality of angles is an equivalence relation.
			\end{enumerate}
		\end{enumerate}
		\item[Exercise:] For various predicates you've encountered, check whether reflexive, symmetric or transitive. Examples of predicates include $\neq, <, >, \leq, \geq, \subseteq$
	\end{description}
	

\end{document}