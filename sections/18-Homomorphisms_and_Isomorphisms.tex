\documentclass[../main.tex]
		
		\begin{document}
			\begin{description}
		\item[Task:] Understand the most natural functions between objects in abstract algebra such as semigroups, monoids or groups.
		\item[Definition:] Let $(A, *)$ and $(B, *)$ ve vitg semigroups, monoids or groups. A function $f: A \rightarrow B$ is called a \underline{homomorphism} if $f(x * y) = f(x) * f(y) \: \forall x, y \in A$. In other words, if $f$ is a function that respects (behaves well with respect) to the binary operation.
		\item[Examples:]
		\begin{enumerate}
			\item[]
			\item Consider $(\mathbb{Z}, +, 0)$ and $(\mathbb{R}^*, \times, 1)$. \\
			Pick $a \in \mathbb{R}^*$, then $f(n) = a^n$ is a homomorphism between $(\mathbb{Z}, +, 0)$ and $(\mathbb{R}^*, \times, 1)$ because $(\mathbb{R}^*, \times, 1)$ is a group, and we proved for groups that $a^{m+n} = f(m+n) = a^m * a^n = f(m) * f(n) \hspace{10mm} \forall m, n \in \mathbb{Z}$.
			\item More generally, $\forall a \in A$ invertible, where $(A, *)$ is a monoid with identity element $e$, $f(n) = a^m$ gives a homomorphism between $(\mathbb{Z}, +, 0)$ and $(A', *, e)*$, where as before $A' = \{a^m \mid m \in \mathbb{Z}\} \subset A$. \\
			We get even better behaviour if we require $f:A \rightarrow B$ to be bijective.
		\end{enumerate}
		\item[Definition:] Let $(A, *)$ and $(B, *)$ both be semigroups, monoids or groups. A function $f:A \rightarrow B$ is called an isomorphism if  $f:A \rightarrow B$ is both bijective \underline{AND} a homomorphism.
		\item[Examples:]
		\begin{enumerate}
			\item[]
			\item Let $A' = \{2^m \mid m \in \mathbb{Z}\} = \{1, 2, \frac{1}{2}, 4, \frac{1}{4}, \dots\}$ \\
			$f(m) = 2^m$ from $(\mathbb{Z}, +, 0)$ to $(A', \times, 1)$ is an isomorphism since $2^m \neq 2^n$ if $m \neq n$.
			\item Let $A' = \{(-1)^m \mid m \in \mathbb{Z} \} = \{-1, 1\}$ \\
			$f(m) = (-1)^m$ from $(\mathbb{Z}, +, 0)$ to $(A', \times, 1)$ is \underline{NOT} an isomorphism since it's not injective $(-1)^2 = (-1)^4 = 1$.
		\end{enumerate}
		\item[Theorem:] Let $(A, *)$ and $(B, *)$ both be semigroups, monoids or groups. The inverse $f^{-1}: B \rightarrow A$ of any isomorphism  $f: A \rightarrow B$ from $A$ to $B$ is itself an isomorphism.
		\item[Proof:] If $f: A \rightarrow B$ is an isomorphism $\Rightarrow f:A \rightarrow B$ is bijective $\Rightarrow f^{-1}:B \rightarrow A$ is bijective (proven when we discussed functions).
		\item To show $f^{-1} B \rightarrow A$ is a homomorphism, let $u, v \in B$. $\exists x, y \in A$ s.t. $x = f^{-1}(u)$ and $y = f^{-1}(v)$, but then $u = f(x)$ and $v = f(y)$.
		\item Since $f: A \rightarrow B$ is a homomorphism, $f(x * y) = f(x) * f(y) = u*v$. Then $f^{-1}(u*v) = f^{-1}(f(x*y)) = x*y = f^{-1}(u)*f^{-1}(v)$ as needed.
		\item[qed]
		\item[Definition:] Let $(A, *)$ and $(B, *)$ both be semigroups, monoids or groups. If $\exists f:A \rightarrow B$ an isomorphism betwen $A$ and $B$, then $(A, *)$ and $(B, *)$ are said to be isomorphic.
		\item[Remark:] "Ismorphic" comes from "iso" same + "morph\'{e}" form same abstract algebra structure on both $(A, *)$ and $(B, *)$ given to you in two different guises. As the French would say: "M\^{e}me Marie, autre chapeau" same Mary, different hat.
	\end{description}
	

\end{document}