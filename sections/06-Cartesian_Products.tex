\documentclass[../main.tex]
		
		\begin{document}
			\begin{description}
		\item[Task:] Understand sets like $\mathbb{R}^1$ in a more theoretical way.
		\item[Recall from Calculus:]
		\item $\mathbb{R} = \mathbb{R}^1 \ni x$
		\item $\mathbb{R} \times \mathbb{R} = \mathbb{R}^2 \ni (x_1, x_1)$ \\ \vdots
		\item $\underbrace{\mathbb{R} \times \mathbb{R}}_{\text{n times}} = \mathbb{R}^n \ni (x_1, x_2, \dots, x_n)$
		\item These are examples of Cartesian products.
		\item[Definition:] Let $A, B$ be sets. The Cartesian product denoted by $A \times B$ consists of all ordered pairs $(x, y) \: s.t. \: x \in A \land y \in B$, \textbf{i.e.} $A \times B = \{(x, y) \mid x \in A \land y \in B \}$
		\item[Further Examples:]
		\begin{enumerate}
			\item[]
			\item $A = \{1, 3, 7\}$ \\
			$B = \{1, 5\}$ \\
			$A \times B = \{(1, 1), (1, 5), (3, 1), (3, 5), (7, 1), (7, 5)\}$ \\
			\textbf{NB:} The order in which elements in a pair matters: $(7, 1)$ is different from $(1, 7)$. This is why we call $(x, y)$ an \underline{ordered} pair.
			\item $A = \{(x, y) \in \mathbb{R}^2 \mid x^2 + y^2 = 1 \} \leftarrow$ circle of radius 1 \\
			$B = \{z \in \mathbb{R} \mid -2 \leq z \leq 2 \} = \{-2, 2\} \leftarrow$ closed interval \\
			$A \times B \leftarrow$ cylinder of radius 1 and height 4
		\end{enumerate}
	\end{description}
	
	\subsection{Cardinality (number of elements) in a Cartesian product}
	If $A$ has $n$ elements and $B$ has $p$ elements, $A \times B$ has $np$ elements.
	\pagebreak
	\begin{description}
		\item[Example:]
		\begin{enumerate}
			\item[]
			\item $ \#(A) = 3 \hspace{10mm} A = \{1, 3, 7\}$ \\
			$\#(B) = 2 \hspace{10mm} B = \{1, 5\}$ \\
			$\#(A \times B) = 3 \times 2 = 6$
			\item Both $A$ and $B$ are infinite sets, so $A \times B$ is infinite as well.
		\end{enumerate}
		\item[Remark:] We can define Cartesian products of any length, \textbf{e.g.} $A \times A \times B \times A$, $B \times A \times B \times A \times B$, etc. If all sets are finite, the number of elements is the product of the numbers of elements of each factor. If $\#(A) = 3$ and $\#(B) = 2$ as above, $\#(A \times B \times A) = 3 \times 3 \times 3 = 27$ and $\#(B \times A \times B) = 2 \times 3 \times 2 = 12$.
	\end{description}
	

\end{document}