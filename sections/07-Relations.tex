\documentclass[../main.tex]
		
		\begin{document}
			\begin{description}
		\item[Task:] Define subsets of Cartesian products with certain properties. Understand the predicates $"="$ (equality) and other predicates in predicate logic in a more abstract light.
		\item Start with $x=y$. The elements $x$ is some notation $R$ to $y$ (equality in this case). We can also denote it as $xRy$ or $(x, y) \in E$
		\item Let $x, y$ in $\mathbb{R}$, then $E = \{(x, x) \mid x \in \mathbb{R} \} \subset \mathbb{R} \times \mathbb{R}$.
		\item The "diagonal" in $\mathbb{R} \times \mathbb{R}$ gives exactly the elements equal to each other.
		\item More generally:
		\item[Definition:] Let $A, B$ be sets. A subset of the Cartesian product $A \times B$ is called a relations between $A$ and $B$. A subset of the Cartesian product $A \times A$ is called a relations on $A$.
		\item[Remark:] Note how general this definition is. To make it useful for understanding predicates, we will need to introduce key properties relations can satisfy.
		\item[Example:] $A = \{1, 3, 7\} \hspace{10mm} B = \{1, 2, 5\}$
		\item We can define a relation $S$ on $A \times B$ by $S = \{(1, 1), (1, 5), (3, 2)\}$. This means $1S1$, $1S5$ and $3S2$ and no other ordered pairs in $A \times B$ satisfy $S$.
		\item[Remark:] The relations we defined involve 2 elements, so they are often called \underline{binary relations} in the literature.
	\end{description}
	

\end{document}