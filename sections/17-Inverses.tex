\documentclass[../main.tex]
		
		\begin{document}
			\begin{description}
		\item[Task:] Understand what an inverse is and what formal properties it satisfies.
		\item[Definition:] Let $(A, *)$ be a monoid with identity element $e$ and let $a \in A$. An element $y$ of $A$ is called the \underline{inverse} of $x$ if $x*y = y*x = e$. If an element $a \in A$ has an inverse, then $a$ is called \underline{invertible}.
		\item[Examples:]
		\begin{enumerate}
			\item[]
			\item $(\mathbb{R}, +)$ has identity element 0. $\forall x \in \mathbb{R}, (-x)$ is the inverse of $x$ since $x+(-x)=(-x)=x = 0$.
			\item $(\mathbb{R}, \times)$ has identity element 1. $x \in \mathbb{R}$ is invertible only if $x \neq 0$. If $x \neq 0$, the inverse of $x$ is $\frac{1}{x}$ since $x \times \frac{1}{x} = \frac{1}{x} \times x =1$.
			\item $(Mn, *)$ the identity element is $In$. $A \in Mn$ is invertible if $det(A) \neq 0$. $A^{-1}$ the inverse is exactly the one you computed in linear algebra. If $det(a) = 0, A$ is \underline{NOT} invertible.
			\item Given a set $A, (P(A), \cup)$ has $\emptyset$ as its identity element of all the elements of $P(A)$ only $\emptyset$ is invertible and has itself asits inverse: $\emptyset \cup \emptyset= \emptyset \cup \emptyset = \emptyset$
		\end{enumerate}
		\item[Theorem:] Let $(A, *)$ be a monoid. If $a \in A$ has an inverse, then that inverse is unique.
		\begin{description}
			\item[Proof:] By contradiction: Assume not, then $\exists a \in A$ s.t. both $b$ and $c$ in $A$ are its inverses, \textbf{i.e.} $a*b = b*a = e$, the identity element of $(A, *)$ and $a*c = c*a = e$ and $b \neq c$, then $b = b*e=b*(a*c)=(b*a)*c=e*c=c$.
			\item[qed]
		\end{description}
		\item Since every invertible element of $a$ for $(A, *)$ a monoid has a unique inverse, we can denote the inverse by the more standard notation $a^{-1}$.
		\item Next, we need to understand inverses of elements obtained via the binary operation:
		\item[Theorem:] Let $(A, *)$ be a monoid and let $a, b$ be invertible elements of $A$. $a*b$ is also invertible and $a*b^{-1} = b^{-1}*a^{-1}$.
		\begin{description}
			\item[Remark:] You might remember this formula from linear algebra when you looked at the inverse of a product of matrices $AB$.
			\item[Proof:] Let $e$ be the identity element of $(A, *)  a*a^{-1} = a^{-1}*a = e$ and $b * b^{-1} = b^{-1}*b = e$. We would like to show $b^{-1} * a*{-1}$ is the inverse of $a*b$ by computing $(a*b)*(b^{-1} * a^{-1})$ and $(b^{-1} * a^{-1}) * (a*b)$ and showing both are $e$.
			\item $(a*b)*(a^{-1}*b^{-1}) = a*(b*b^{-1})*a^{-1} = a*e*a^{-1} = a*a^{-1} = e$
			\item $(b^{-1}*a^{-1})*(a*b) = b^{-1}*(a^{-1}*a)*b = b^{-1}*e*b = (b^{-1}*e)*b = b^{-1}*b=e$
			\item Thus $b^{-1}*a^{-1}$ satisfies the conditions needed for it to be the inverse of $a*b$.Since an inverse of unique, $a*b$ is invertible and $b^{-1}*a^{-1}$.
		\end{description}
		\item[Theorem:] Let $(A, *)$ be a monoid, and let $a, b \in A$. Let $x \in A$ be invertible. $a=b*x \Leftrightarrow b = a*x^{-1}$. Similarly, $a=x*b \Leftrightarrow b = x^{-1}*a$
		\begin{description}
			\item[Proof:] Let $e$ be the identity element of $(A, *)$.
			\item First $a=b*x \Leftrightarrow b = a*x^{-1}$:
			\item $\Rightarrow$ Assume $a=b*x. a*x^{-1} = (b*x)*x^{-1} = b*{x*x^{-1}} = b*e = b$ as needed.
			\item $\Leftarrow$ Assume $b=a*x^{-1}$. Then $b*x=(a*x^{-1})=a*(x^{-1}*x)=a*e=1$ as needed.
			\item Apply the same type of argument to show $a=x*b \Leftrightarrow b=x^{-1}*a$.
			\item[qed]
		\end{description}
		\item Let $(A, *)$ be a monoid. We can now make sense of $a^n$ for $n \in \mathbb{Z}, n < -$ for every $n \in A$ invertible. Since $n$ is a negative integer, $\exists p \in \mathbb{N}$ s.t. $n=-1$. Set $a^n = a^{-p}=(a^p)^{-1}$.
		\item[Theorem:] Let $(A, *)$ be a monoid and let $a \in A$ be invertible. Then $a^n * a^m = a^{m+n} \hspace{10mm} \forall m, n \in \mathbb{Z}$.
		\begin{description}
			\item[Proof:] When $m \geq 0 \land n \geq n$ we have already proven this result. The rest of the proof splits into cases.
			\item[Case 1:] $m=n \lor n = 0$
			\item If $m=0, n \in \mathbb{Z}, a^m * a^n = a^0 * a^n = e * a^n = a^n = a^{0+n}$ as needed.
			\item If $m \in \mathbb{Z}, n = 0, a^m * a^n = a^m * a^0 = a^m * e = a^m = a^{m+0}$ as needed.
			\item[Case 2:] $m < 0 \land n < 0$
			\item $m < 0 \Rightarrow \exists p \in \mathbb{N} \: s.t. \: p = -m. \: n<0 \Rightarrow \exists q \in \mathbb{N} \: s.t. \: q = -n.$
			\item $a^m = a^{-p} = (a^p)^{-1} \land a^n = a^{-q} = (a^q)^{-1}$
			\item $a^m * a^n = (a^p)^{-1} * (a^q)^{-1} = (a^q * a^p)^{-1} = (a^{p+q})^{-1} = a^{-(p+q)} = a^{-q-p} = a^{m+n} = a^{n+m}$
			\item[Case 3:] $m \land n$ have opposite signs.\\
			Without loss of generality, assume $m<0 \land n>0$ (the case $m>0 \land n<0$ is handled by the same argument). Since $m<0, \exists p \in \mathbb{N} \: s.t. \: p = -m.$ This case splits into two subcases:
			\begin{description}
				\item[Case 3.1:] $m+n \geq 0$
				\item Set $q = m+n$. Then $a^{m+n} = a^q = e*a^q = (a^p)^{-1} * a^p * a^q = (a^p)^{-1} * a^{p+q} = a^{-p} * a^{p+q} = a^m * a^{-m+m+n} = a^m * a^n$
				\item[Case 3.2:] $m+n < 0$
				\item Set $q = -(m+n) = -m-n \in \mathbb{N}^*$. Then $a^{m+n} = a^{-q} = (a^q)^{-1} * e = (a^q)^{-1} * (a^{-n} * a^n) = (a^q)^{-1} * (a^n)^{-1} *a^n = (a^n * a^q)^{-1} * a^n = (a^{n+p})^{-1} * a^n = (a^{n-m-n})^{-1} * a^n = (a^{-m})^{-1} * a^n = (a^p)^{-1} * a^n = a^m * a^n$
			\end{description}
		\end{description}
		\item[Theorem:] Let $(A, *)$ be a monoid, and let $a$ be an invertible element of $A$. $\forall m, n \in \mathbb{Z}, (a^m)^n = a^{mn}$.
		\begin{description}
			\item[Proof:] We consider 3 cases:
			\item[Case 1:] $n > 0$, \textbf{i.e.} $n \in \mathbb{N}^*. \: m \in \mathbb{Z}$ with no additional restrictions we proceed by induction on $m$.
			\begin{description}
				\item[Base Case:] $n=1 \hspace{10mm} (a^m)^1 = a^m = a^{m \times 1}$
				\item[Inductive Step:] We assume $(a^m)^n = a^{mn}$ and seek to prove $(a^m)^{n+1} = a^{m(n+1)}$. Start with $(a^m)^{n+1} = (a^m)^n * (a^m)^1 = a^{mn} * a^m = a^{mn+m} = a^{m(n+1)}$
			\end{description}
			\item[Case 2:] $n=0$; no restriction on $m \in \mathbb{Z}$
			\item $(a^m)^n = (a^m)^0 = e = a^0 = a^{m \times 0} = a^{mn}$
			\item[Case 3:] $n < 0$; no restriction on $m \in \mathbb{Z}$.
			\item Since $n < 0, \exists p \in \mathbb{N} \: s.t. \: p = -n$. By case 1, $(a^m)^p = a^{mp}$
			\item $(a^m)^n = (a^m)^{-p} = ((a^m)^p)^{-1} = (a^{mp})^{-1} = a^{-mp} = ^{m(-p)} = a^{mn}$
		\end{description}
	\end{description}
	
	\subsection{Groups}
	A motion formally defined in the 1870's even though theorems about groups proven as early as a century before that.
	\begin{description}
		\item[Definition:] A group is a monoid in which every element is invertible. In other words, a group is a set $A$ endowed with a binary operation $*$ satisfying the following properties:
		\begin{enumerate}
			\item $*$ is associative, \textbf{i.e.} $\forall x, y, z \in A, (x*y)*z = x*(y*z)$
			\item There exists an identity element $e \in A$, \textbf{i.e.} $\exists e \in A \: s.t. \: \forall a \in A, a*e = e*a = a$
			\item Every element of $A$ in invertible, \textbf{i.e.} $\forall a \in A \exists a^{-1} \in A \: s.t. \: a * a^{-1} = a^{-1} * a = e$
		\end{enumerate}
		\item[Notation for Groups:] $(A, *) \lor (\underset{set}{\underbrace{A}}, \underset{operation}{\underbrace{*}}, \underset{identity}{\underbrace{e}})$
		\item[Remark:] Closure under the operation $*$ is \underline{implicit} in the definition \textbf{i.e.} $\forall a, b \in A, a * b \in A$
		\item[Definition:] A group $(A, *, e)$ is called \underline{commutative} or \underline{Abelian} if its operation $*$ is commutative.
		\item[Examples:]
		\begin{enumerate}
			\item[]
			\item $(\mathbb{R}, +, 0)$ is an Abelian group.
			\item[] $-x$ is the inverse of $x, \forall x \in \mathbb{R}$
			\item $(\mathbb{Q}^*, \times, 1) \hspace{10mm} \mathbb{Q}^* = \mathbb{Q}^*\backslash\{0\} \hspace{10mm} (\mathbb{Q}^*, \times, 1)$ is Abelian
			\item[] $\forall q \in \mathbb{Q}^*, q^{-1} = \frac{1}{q}$ is the inverse.
			\item $(\mathbb{R}^3, +, 0)$ vectors in $\mathbb{R}^3$ with vector addition forms an Abelian group.
			\item[] $(x, y, z)+(x', y', z') = (x+x', y+y', z+z')$ vector addition.
			\item[] $0 = (0, 0, 0)$ is the identity. \hspace{10mm} $(-x, -y, -z) = - (x, y, z)$ is the inverse of $(x, y, z)$.
			\item $($\~{M}$m, *, In)$ $n\times n$ invertible matrices with real coefficients under matrix multiplication with $In$ as the identity elements forms a group which is \underline{NOT} Abelian.
			\item Set $A = \mathbb{Z}$ and recall the equivalence relation $x \equiv y$ mod $3$ \textbf{i.e.} $x \land y$ have te same remainder under the division by $3$. Recall that $\mathbb{Z} / N = \{0, 1, 2\}$, \textbf{i.e.} the set of equivalence classes under the partition determined by this equivalence relation. We denote $\mathbb{Z} / N \ \{0, 1, 2\} = \mathbb{Z}_3$
			\item[] Consider $(\mathbb{Z}_3, \oplus _3, 0)$ where $\oplus _3$ is the operation of addition modulo $3$, \textbf{i.e.} $1+0=1, 1+1=2, 1+2=3 \equiv 0$ mod $3$.
			\item[\textbf{Claim:}] $(\mathbb{Z}_3, \oplus _3, 0)$ is an Abelian group.
			\item[\textbf{Proof of Claim:}] Associativity of $\oplus _3$ follows from the associativity of $+$, addition of $\mathbb{Z}$. Clearly, $0$ is the identity (don't forget $0$ stands for all elements with remainder $0$ under division by $3$, \textbf{i.e.} $\{0, 3, -3, 6, -3, \dots\}$). To compute inverses recall that $a \oplus _3 a^{-1} = 0, 0$ is the inverse of $0$ because $0+0=0$. $2$ is he inverse of $1$ because $1+2=3 \equiv 0$ mod $3$, and $1$ is the inverse of $2$ because $2+1=3 \equiv 0$ mod $3$. \\
			More generally, consider the equivalence relation on $\mathbb{Z}$ given by $x \equiv y$ mod $n$ for $n \geq 1. \mathbb{Z} / N = \{0, 1, \dots, n-1\} = \mathbb{Z}_n$. All possible remainders under division by $n$ are the equivalence classes. Let $\oplus _n$ be addition mod $n$. By the same argument as above, $(\mathbb{Z}_n, \oplus _n, 0)$ is an Abelian group.
			\item[\textbf{Q:}] What is we consider multiplication mod $n$, \textbf{i.e.} $\otimes _n$. Is $(\mathbb{Z}_n, \otimes _n, 1)$ a group?
			\item[\textbf{A:}] No! $(\mathbb{Z}_n, \otimes _n, 1)$ is not even a monoid because $1 \otimes _n 0 = 0 \otimes _n 1 = 0$, so $1$ is not an identity element for $\otimes _n$ on $\mathbb{Z}_n$.
			\item[\textbf{Q:}] Can this be fixed?
			\item[\textbf{A:}] Troubleshoot how to get rid of 0. \\
			Consider $\mathbb{Z}_n^* = \mathbb{Z}_n \backslash \{0\} = \{1, 2, \dots, n-1\}$ all non-zero elements in $\mathbb{Z}_n^*$. This eliminates 0 as an element, but can 0 arise any other way from the binary operation? It turns out the answer depends on $n$. If $n$ is not prime, say $n=6$, we get two divisors, \textbf{i.e.} elements that yield 0 when multiplied by \underline{precisely} the factors of $n$, for $n=6$, $\mathbb{Z}^*_6 = \{1, 2, 3, 4, 5\}$ \underline{but} $2 \otimes _6 3 = 6 \equiv 0$ mod $6$, so $2 \land 3$ are two divisors.
			\item[\textbf{Claim:}] If $n$ is prime, then $(\mathbb{Z}_n^*, \otimes _n, 1)$ is an Abelian group. \\
			Used in cryptology $\rightarrow$ you will see next semester. \\
			As an example, let us look at the multiplication table for $\mathbb{Z}_5^*$ to see the inverse of various elements: $\mathbb{Z}^*_5 = \mathbb{Z}_5 \backslash \{0\} = \{0, 1, 2, 3, 4,\} \backslash \{0\} - \{1, 2, 3, 4\}$
	
			\begin{center}
				\begin{tabular}[h!]{|c|cccc|rl}
					\cline{1-5}
					& 1 & 2 & 3 & 4\\ \cline{1-5}
					1 & 1 & 2 & 3 & 4 & \hspace{10mm} $1^{-1} = 1$ & $1 \otimes_5 1 = 1$ \\
					2 & 2 & 4 & 1 & 2 & \hspace{10mm} $2^{-1} = 3$ & $2 \otimes_5 3 = 6 \equiv 1$ mod $5$ \\
					3 & 3 & 1 & 4 & 2 & \hspace{10mm} $3^{-1} = 2$ & $3 \otimes_5 2 = 6 \equiv 1$ mod $5$ \\
					4 & 4 & 3 & 2 & 1 & \hspace{10mm} $4^{-1} = 1$ & $4 \otimes_5 4 = 16 \equiv 1$ mod $5$ \\
					\cline{1-5}
				\end{tabular}
			\end{center}
			The fact that $\mathbb{Z}^*_n, \otimes _n, 1$ is Abelian follows from the commutativity of multiplication on $\mathbb{Z}$.
			\item Let $(A, *, e)$ be any group and let $a \in A$. \\
			Consider $A' = \{a^m \mid m \in \mathbb{Z}\}$ all powers of $a$. It turns out $(A', *, e)$ is a group called the \underline{cyclic group} determined by $a$. $(A', *, e)$ is Abelian \underline{regardless} of whether the original group was Abelian or not because of the theorem we proved on powers of $a$: $\forall m, n \in \mathbb{Z} \hspace{10mm} a^m * a^n = a^{m+n} = a^{n+m} = a^n * a^m$. \\
			Cyclic groups come in two flavours: finite ($A'$ is a finite set) and infinite ($A'$ is an infinite set). \\
			For example, let $(A, *, e) = (\mathbb{Q}^*, , \times , 1)$ \\
			If $a=-1 \hspace{10mm} A' = \{(-1)^m \mid m \in \mathbb{Z}\} = \{-1, 1\}$ is finite. \\
			If $a=2 \hspace{10mm} A' = \{2^m \mid m \in \mathbb{Z}\} = \{1, 2, \frac{1}{2}, 4, \frac{1}{4}, \dots\}$ is infinite.
		\end{enumerate}
	\end{description}
	

\end{document}